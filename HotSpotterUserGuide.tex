\documentclass[a4paper,10pt]{article}
\usepackage[utf8]{inputenc}
\usepackage{graphicx}
\usepackage{enumerate}
\usepackage{listings}
\usepackage{multirow}
\usepackage{pdfpages}
\usepackage{amsmath}
\usepackage{amsthm}
\usepackage[margin=1.3in]{geometry}
\usepackage[backref=true,pagebackref=true,breaklinks=true,letterpaper=true,colorlinks,bookmarks=true,bookmarksnumbered=true,bookmarksopen=true,hyperfigures=true]{hyperref}
\usepackage{times}
\usepackage{textcomp}
\usepackage{fancyvrb}
% Uncomment to disallow hyphenated line breaks
\usepackage[none]{hyphenat}
\raggedright


\newcommand{\HRule}{\rule{\linewidth}{0.5mm}}
\newcommand{\eps}{\varepsilon}
\newcommand{\ssum}{\displaystyle\sum}

% MACRO DEFINITIONS
\newcommand{\developeremail}{{\tt hotspotter.ir@gmail.com}}


\begin{document}
\begin{titlepage}
    \topskip0pt
    \vspace*{\fill}

    \begin{center}

        \HRule \\[0.4cm]
        { \huge \bfseries HotSpotter User Guide}\\[0.4cm]
        \HRule \\[1.5cm]

        Jon Crall (crallj@rpi.edu)\\
        Jason Parham (parhaj@rpi.edu) \\
        Chuck Stewart (stewart@cs.rpi.edu) \\ 

        \medskip

        Department of Computer Science \\
        Rensselaer Polytechnic Institute

    \end{center}

    \vspace*{\fill}
\end{titlepage}

\DeclareRobustCommand{\cmdkey}{\raisebox{-.035em}{\includegraphics[height=.75em]{images/command}}}

\section{Usage}  This document assumes that you have already downloaded and
installed the Windows or Mac version of the HotSpotter software.  The software
is compatible with Linux, but is distributed as source only; installers are not
currently provided for Linux. Furthermore, this document's purpose is to describe the basic steps of running HotSpotter to
identify individual animals, including opening a database (or creating a new
one), importing images, selecting ROIs, querying, and naming.  The instructions
are primarily focused on the Mac version of the software, but adaptation to the
Windows version is easy.  Just know that the control key ({\tt Ctrl}) on Windows
is equivalent to the command key ({\tt Cmd} = \cmdkey) on Mac. This guide will
use the Mac notation.

    \subsection{Opening the Program}
    When HotSpotter is first run, the program prompts the user to open a
    database or create a new one.  In each succeeding run, it will start by
    opening the previous database. 

    % Macros for displaying Mac/Windows Control Keys and arrows instead of ->
    \renewcommand{\textrightarrow}{$\rightarrow$}
    %\newcommand{\textcmd}{Cmd}
    \newcommand{\textcmd}{\cmdkey}
    %[fontsize=\footnotesize,frame=single,commandchars=\\\{\}]

    To create a new database: \\
    \begin{Verbatim}[commandchars=\\\{\}]
    File \textrightarrow New Database [\textcmd+N]
    \end{Verbatim}

    \begin{center}
        \includegraphics[scale=0.15]{images/start.png}
    \end{center}

    To open an existing database
    \begin{Verbatim}[commandchars=\\\{\}]
    File \textrightarrow Open Database [\textcmd+O]
    \end{Verbatim}


    HotSpotter can also read StripeSpotter databases by opening the
    StripeSpotter database's {\tt data} directory.  Previous versions of HotSpotter databases
   are also compatible.
    
    \subsection{Importing images}
        In order to add one or more images to the database, there are two options. To manually select specific images from a directory, there is the command
        \begin{Verbatim}[commandchars=\\\{\}]
        File \textrightarrow Import Images (select file(s)) [\textcmd+I]
        \end{Verbatim}
        There is also the option to import an entire directory of images directly into HotSpotter.  This may be achieved with the command 
        \begin{Verbatim}[commandchars=\\\{\}]
        File \textrightarrow Import Images (select directory)
        \end{Verbatim}
        \begin{center}
            \includegraphics[scale=0.15]{images/added.png}
        \end{center}
        HotSpotter will copy all selected images into the current database's directory; images that have been added may be seen under the ``Image Table'.

    \subsection{Defining Chips with ROIs and Orientation}
        Before using HotSpotter to identify an animal in an image ---
        or, equivalently, find other images that show the same
        animal ---- a region of interest (ROI) and an orientation must
        be assigned.  (The sub-image extracted from an ROI is called a
        ``chip''.)  The ROIs must be specified first, one for each
        animal you would like to have HotSpotter identify.
%        I REMOVED THIS NEXT PARAGRAPH.  
%        This is accomplished by
%        \begin{Verbatim}[commandchars=\\\{\}]
%        Help \textrightarrow Convert All Images to Chips
%        \end{Verbatim}
%        This option should be used in the relatively rare case that the
%        animal occupies almost the entire image.
%
%        The more common case is to specify the ROIs manually.  Multiple ROIs are allowed for each
%         image.  
        Each ROI should include most of the body of the animal ---
        anything that might be a distinguishing feature --- so users should err on the
        side of making the ROI too large rather than too small.

        In order to specify an ROI, the Image Table should be
        highlighted and an image should be selected.  Next, click
        \begin{Verbatim}[commandchars=\\\{\}]
        Actions \textrightarrow Add Chip [A]
        \end{Verbatim}
        and select the ROI by clicking two points in the resulting Image
        View'to specify opposite corners of the ROI bounding box.

        \;

        If you do not like the resulting ROI, you can click on it to reselect an new ROI for that chip by
        using
        \begin{Verbatim}[commandchars=\\\{\}]
        Actions \textrightarrow Reselect ROI [R]
        \end{Verbatim}
        You can use remove it entirely by using 
        \begin{Verbatim}[commandchars=\\\{\}]
        Actions \textrightarrow Delete Chip
        \end{Verbatim}

        The default orientation of every chip is horizontal, and this is set
        automatically by HotSpotter.  This is usually sufficient when
        taking ``normal'' pictures of standing
        animals, such as zebras or giraffes.  On the other hand, for overhead
        pictures of animals like frogs, specification of the
        orientation is \textbf{crucial for accurate recognition}.  The
        orientation is best determined by drawing an axis within the
        ROI of the animal in a way that can be repeated for each
        animal.  For frog images, the repeatable orientation is selected along the spine, from the tip of the mouth to tip of the tail.  For zebra images, the repeatable orientation is 
	selected along the top of the back, from the withers to the point of the croup.  Any orientation can be used, as long as it is repeatable and can be generally applied to all members of the species.

 	\;
        
	In order to specify an orientation other than the default
        (horizontal) orientation, the user must click
        \begin{Verbatim}[commandchars=\\\{\}]
        Actions \textrightarrow Reselect Orientation [O]
        \end{Verbatim}
        \begin{center}
            \includegraphics[scale=0.13]{images/image.png}
        \end{center}

        \noindent
        By clicking on two points within the Image View, the chip will be given a new orientation along the selected
       orientation axis.  Note that the angle does not have to be
        selected perfectly each time --- pretty close will suffice ---
        but you should be consistent with the order in which you
        click (for example, with frogs, by selecting the mouth and then subsequently the tail).  By reversing the order
	of the clicks, you will orient the chip to be inverted compared to all other chips, which will cause HotSpotter to fail.

    \subsection{Chip Properties Display (Optional)} 
        HotSpotter uses each ROI and orientation to generate a chip. 
        Within these chips HotSpotter computes its hotspots --- elliptical
        regions centered on points of interest that HotSpotter automatically
        detects.  Intuitively, the hotspots are loosely analogous to a
        part of a ``fingerprint'' for the chip.  Two chips having enough hotspot similarity
        will be matched successfully by HotSpotter. A chip can be seen by clicking 
	on the Chip Table and then selecting a chip.

        \begin{center}
          \includegraphics[scale=0.2]{images/chip.png}
        \end{center}

        The hotspots' points of interest and elliptical regions can be
        toggled on and off by clicking on the grey area around a chip in the Chip View.
	A specific hotspot can be viewed by clicking on a point of interest on the chip within the Chip View.
        \begin{center}
            \includegraphics[scale=0.1]{images/chip-ellipse.png}
            \includegraphics[scale=0.1]{images/chip-hotspot.png}
        \end{center}

    \subsection{Running a Query}
        A Query can be run on any selected chip.

        \begin{Verbatim}[commandchars=\\\{\}]
        Actions \textrightarrow Query [Q]
        \end{Verbatim}
        
\noindent
        This will quickly find similar chips in the database.  HotSpotter will
        automatically rank the chips in order of similarity and will highlight
        the portions of the image that it identifies as being most
        similar.

        \begin{center}
            \includegraphics[scale=0.2]{images/query-all.png}

        \;

            \includegraphics[scale=0.212]{images/query-result.png}
        \end{center}
        In this example Query View window, there are six results, each showing a pair of images.  In each
        pair, the query chip is shown on top and the
        potentially-matching chip is shown on the bottom.  Clicking on
        the pair will show a highlighted display.  You will also
        notice a score for each match.  The scores tend to vary according to species and size of the image set.  
	Small image sets will produce greater scores as will certain datasets, such as giraffes and Grevy's
        zebras. 
     
\noindent
  Once you decide, based on the query, which chips show the
  same animal, you may record this decision by giving the chips the same
  name. This requires an understanding
  of the meaning of the names within HotSpotter, as described next.

\subsection{IDs, Names, and Recording Matching Results}

   In the Chip Table, users will see Chip ID, Name, and Image Name columns.

\begin{center}
            \includegraphics[scale=0.1]{images/names.png}
            \includegraphics[scale=0.1]{images/matches.png}
\end{center}

\noindent
  The Chip ID is the unique numerical index that HotSpotter applies to
  each chip.  (Remember, there can be more than one ROI/chip per image.)
  As HotSpotter does its work and chips are successfully matched,
  users should assign unique names to individual animals and use the same 
  name for all chips in which that specific animal appears.  Initially,
  before an image is recognized, the ``Name'' column value will
  be specified as ``\_\_\_\_'' (four underscores).  This denotes an
  unidentified chip - a user may double-click on a name to edit. 
  To view animals that match, simply go to the Name Table to view all of the matched animals with the same name. 
  Use of copying and pasting from one chip name to the other makes this process less prone to
  typing errors.



\section{Additional Tools and Tricks}


Here is a brief discussion of are a few additional tricks and options
for running HotSpotter:
\begin{itemize}
\item \verb+Actions -> Select Next+:
    selects the next image in the database. 
    %selects either the next image that does not already have an ROI or the next chip without an
    %orientation. 

\item \verb+Actions -> New Chip Property+:  
    record metadata as a series of one or more attribute/value pairs for any
    user defined metadata.  HotSpotter will automatically import existing
    metadata from StripeSpotter databases.


%\item \verb+Options -> Toggle Plot Widget+: 
    %shows the results plot in a separate pane.  This is particularly useful for
    %resizing the pane when there are many results.

\item \verb+Options -> Edit Preferences+: 
    change the behavior of HotSpotter. For now, these are not very well
    documented and should only be used with specific guidance from 
    the HotSpotter team.

%\item \verb+Help -> Convert All Images to Chips+: 
    %make each image its own ROI and therefore image chip.

%\item \verb+Help -> Batch Change Name+:  
    %change all chip instances of a given name to a new name. This is useful for
    %when the same animal is grouped under two different names.

%\item \verb+Help -> Assign Matches Above Threshold+: 
    %HotSpotter will automatically run each chip in the database as a query and
    %assigns it as a match to any chips whose matching score is above a
    %user-define threshold. 

\item \verb+Help -> View Data Directory+: 
    Opens the current database directory.

\item \verb+Help -> View Source Directory+: (primarily for developer usage)
    Opens the HotSpotter source directory. 

\item \verb+Help -> View Internal Directory+: 
    Opens the current database's {\tt \_hsdb} directory

\item \verb+Help -> Delete Computed Directory+: 
    Removes all of HotSpotter's internal cache. It should be used if you suspect
    the database has been corrupted. This forces everything to be recomputed. 

\item \verb+Help -> Delete Global Preferences+: 
    This resets hotspotters external preferences. You may need to do this if you
    are upgrading versions. 
    %This removes HotSpotter's external cache, which may have inconsistent
    %file formats between versions. If HotSpotter has trouble loading try this
    %first. 

%\item \verb+Matching Experiment+: (primarily for developer usage)
    %Runs an experiment to see what matches HotSpotter assigns to each chip.
    %Output is written to the database directory. 

%\item \verb+Run Name Consistency Experiment+: (primarily for developer usage) 
    %Runs an experiment to see if HotSpotter agrees with the current labeling.
    %Output is written to the database directory. 

\end{itemize}
  


\section{A Bit of Troubleshooting}

In the event that HotSpotter behaves unexpectedly, the first thing to try is a
restarting of the program. If the error persists, the following will fix common
errors: 

\begin{itemize}
    \item \textbf{Delete your preference directory.}\\
        HotSpotter keeps a small set of preference files in the user's  home directory.
        These files remember the last database opened as well as other
        preferences. When updating to new versions these can sometimes cause
        problems. Deleting the {\tt \texttildelow/.hotspotter}\footnote{Note
            that {\tt \texttildelow} denotes the user's home folder} folder may fix some issues.
            
        You may also perform this action from HotSpotter by using: \verb+Help -> Delete Global Preferences+: \\

    %\item \textbf{Re-Import the Images}\\
        %If the images you've imported aren't showing up, you can always re-import
        %the images in\\ {\tt user\_database\_dir/images}
        %directory. (REALLY???  THIS DOES NOT CREATE PROBLEMS???)  \\

    \item \textbf{Delete the Computed Directory}\\
        If something looks corrupted or ROIs are being oddly drawn, the user
        should consider deleting the computed directory.  Running  the command 
        {\tt (Help \textrightarrow{} Delete Computed Directory)} will delete cached and 
	other temporary data and force HotSpotter to recompute all of its data; this can be done safely, but at the cost of 
	higher initial wait times as HotSpotter recomputes some information. \\


    \item \textbf{Mac OSX 10.8 Mountrain Lion \& 10.9 Mavericks Gatekeeper}\\
      On the latest two Mac operating systems, the software may not run immediately.
      This error is due to a security feature called Gatekeeper.  If the app
      fails to run, please do the following:
        \begin{enumerate}
            \item Go to {\tt System Preferences} --- Click the Apple icon in the menu bar (top-left of the screen) and select {\tt System Preferences} in the drop down menu.
            \item Go to {\tt Security \& Privacy} --- It is located on the top row, entitled {\tt Personal}.
            \item Go to the {\tt General} tab.
	  \item On 10.9 Mavericks, there may be a button that will allow HotSpotter to open.  Simply click this button, skip the remaining steps, and attempt to open up HotSpotter again.  For 10.8 Mountain Lion users, continue below.
            \item Authenticate --- Click on the lock at the bottom-left corner of the screen and subsequently input your computer username and password.
            \item In the bottom half of the {\tt General} tab, there will be the following selection:
                \begin{Verbatim}
                    Allow applications downloaded from:
                    ( ) Mac App Store
                    (X) Mac App Store and identified developers
                    ( ) Anywhere

                \end{Verbatim}

                Select {\tt Anywhere} and subsequently select {\tt Allow From Anywhere} in the drop down warning.

            \item Close the {\tt System Preferences} window.
            \item Install HotSpotter and run it.
            \item To re-enable security after running HotSpotter once,
                repeat the above changes to your preferences, except click on
                \texttt{Mac App Store and identified developers}.
        \end{enumerate}


    \item \textbf{Email the Developer}\\
        If all else fails users should send an email to \developeremail{}.
        Please include a detailed description of the error and what was being done when it
            happened.  Then also please copy the text in the output window to a text file, and include this in your email.\\
    \end{itemize}

%\section{Source Code Dependencies}
%
%The remainder of this discussion only applies to downloading and
%working with the source code instead of the installer packages.
%
%Before executing HotSpotter from the source code users should ensure
%that their environment is set up correctly. Primarily, this includes
%Python 2.7.3, Qt, and OpenCV, but it also includes several supporting
%packages.  Users who want to use HotSpotter without modification
%should download the installer package instead.
%
%    \subsection{Windows}
%        Install the following dependencies in order.  
%        \textbf{The software is untested using 64-bit python. It is preferred to use 32-bit builds of each dependency when specified.}
%        \begin{enumerate}
%            \item Python 2.7 32-bit
%                \begin{enumerate}
%                    \item Download:
%                        \url{http://www.python.org/download/releases/2.7.5/}
%                    \item Install with install packager
%                \end{enumerate}
%
%            \item MinGW (C / C++) 
%                \begin{enumerate}
%                    \item Download: 
%                        \url{http://sourceforge.net/projects/mingw/files/latest/download?source=files}
%                    \item Install with install packager
%                    \item REQUIRED: C and C++ COMPILER
%                \end{enumerate}
%
%            \item Qt Library 4.8 32-bit 
%                \begin{enumerate}
%                    \item Download: \url{http://qt-project.org/downloads}
%                    \item Install with install packager
%                \end{enumerate}
%
%            \item PyQt4 32-bit 
%                \begin{enumerate}
%                    \item Download:
%                        \url{http://www.riverbankcomputing.com/software/pyqt/download}
%                    \item Install with install packager
%                \end{enumerate}
%
%            \item NumPy 32-bit 
%                \begin{enumerate}
%                    \item Download:
%                        \url{http://sourceforge.net/projects/numpy/files/NumPy/1.7.1/numpy-1.7.1-win32-superpack-python2.7.exe/download}
%                    \item Install with install packager
%                \end{enumerate}
%
%            \item matplotlib 32-bit 
%                \begin{enumerate}
%                    \item Download: \url{http://matplotlib.org/downloads.html}
%                    \item Install with install packager
%                \end{enumerate}
%
%            \item PIL 32-bit 
%                \begin{enumerate}
%                    \item Download:
%                        \url{http://www.pythonware.com/products/pil/}
%                    \item Install with install packager
%                \end{enumerate}
%        \end{enumerate}
%        \;
%
%    \subsection{Mac OSX}
%
%        Install the following dependencies in order.
%        \begin{enumerate}
%
%            \item Qt Library 4.8 
%                \begin{enumerate}
%                    \item Download: \url{http://qt-project.org/downloads}
%                    \item Install with install packager
%                \end{enumerate}
%
%            \item XQuartz - This dependency is required for FreeType
%                (\url{http://www.freetype.org/}) and libpng
%                (\url{http://www.libpng.org/pub/png/libpng.html}) 
%                \begin{enumerate}
%                    \item Download: \url{http://xquartz.macosforge.org}
%                    \item Install with install packager
%                \end{enumerate}
%
%            \item SIP 
%                \begin{enumerate}
%                    \item Download:
%                        \url{http://riverbankcomputing.co.uk/software/sip/download}
%                    \item sudo python configure.py
%                    \item sudo make
%                    \item sudo make install
%                \end{enumerate}
%
%            \item PyQt4 
%                \begin{enumerate}
%                    \item Download:
%                        \url{http://www.riverbankcomputing.com/software/pyqt/download}
%                    \item sudo python configure.py
%                    \item sudo make
%                    \item sudo make install
%                \end{enumerate}
%
%            \item NumPy 
%                \begin{enumerate}
%                    \item Download:
%                        \url{http://sourceforge.net/projects/numpy/files/latest/download?source=files}
%                    \item sudo python setup.py install
%                \end{enumerate}
%
%            \item matplotlib 
%                \begin{enumerate}
%                    \item Download: \url{http://matplotlib.org/downloads.html}
%                    \item sudo python setup.py install
%                \end{enumerate}
%
%            \item libjpeg 
%                \begin{enumerate}
%                    \item Download: \url{http://www.ijg.org/}
%                    \item ./configure
%                    \item sudo make
%                    \item sudo make install
%                \end{enumerate}
%
%            \item PIL 
%                \begin{enumerate}
%                    \item Download:
%                        \url{http://www.pythonware.com/products/pil/}
%                    \item sudo python setup.py install
%                \end{enumerate}
%        \end{enumerate}
%        \;
%

\section{Source Code}

    \subsection{License}
HotSpotter is currently distributed under the Apache License, Version 2.0.
\begin{Verbatim} 
   HotSpotter
   Copyright © 2014 Jon Crall, Jason Parham, Chuck Stewart
   Department of Computer Science 
   Rensselaer Polytechnic Institute

   Licensed under the Apache License, Version 2.0 (the "License");
   you may not use this file except in compliance with the License.
   You may obtain a copy of the License at

       http://www.apache.org/licenses/LICENSE-2.0

   Unless required by applicable law or agreed to in writing, software
   distributed under the License is distributed on an "AS IS" BASIS,
   WITHOUT WARRANTIES OR CONDITIONS OF ANY KIND, either express or implied.
   See the License for the specific language governing permissions and
   limitations under the License.


\end{Verbatim}

\subsection{Download}
Download the source code here: \url{https://github.com/Erotemic/hotspotter}
\begin{Verbatim}[commandchars=\\\{\}]
git clone git@github.com:Erotemic/hotspotter.git
\end{Verbatim}

%Users will also need to check out the tpl submodule.  This can be be
%done separately, or by running the command: 
%\begin{Verbatim}[commandchars=\\\{\}]
%python setup.py configure
%\end{Verbatim}
%This will also ensure that files have the correct permissions. 

Once the source code has been downloaded the program can be run by using the command:
\begin{Verbatim}[commandchars=\\\{\}]
./main.py
\end{Verbatim}

        %To build the program into a Windows .exe, execute the command:
        %\begin{Verbatim}
        %python setup.py py2exe
        %\end{Verbatim}

        %To build the program into a Mac .app, execute the command:
        %\begin{Verbatim}
        %python setup.py py2app
        %\end{Verbatim}

\subsection{Contribute}
HotSpotter is an open source project. If any tech-savvy users develop a cool
feature or a bug-fix and would like to see it incorporated, send an email with the proposed
patch to \developeremail{} for code review.

\newpage



\end{document}
