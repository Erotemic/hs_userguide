
\newcommand{\HRuled}{\rule{\linewidth}{0.1mm}}

    \textbf{Mac OSX 10.9 Mavericks --- Gatekeeper}\\
      On the latest two Mac operating systems, the software may not run immediately. 
      This error is due to a security feature called Gatekeeper.  If the app
      fails to run, please do the following:
        \begin{enumerate}
            \item Go to {\tt System Preferences} --- Click the Apple icon in the menu bar (top-left of the screen) and select {\tt System Preferences} in the drop down menu.
            \item Go to {\tt Security \& Privacy} --- It is located on the top row, entitled {\tt Personal}.
            \item Go to the {\tt General} tab.
	  \item On 10.9 Mavericks, there may be a button that will allow HotSpotter to open.  
	\begin{Verbatim}
   "HotSpotter" was blocked from opening because it is not from an 
   identified developer
          \end{Verbatim}

	Simply click the button {\tt Open anyway}.  You may stop here, skip the remaining steps, and continue to use HotSpotter. 

           \HRuled

	\item Authenticate --- If the {\tt Open anyway} button did not appear, click on the lock at the bottom-left corner of the screen and subsequently input your computer username and password.
            \item In the bottom half of the {\tt General} tab, there will be the following selection:
                \begin{Verbatim}
   Allow applications downloaded from:
   ( ) Mac App Store
   (X) Mac App Store and identified developers
   ( ) Anywhere
                \end{Verbatim}

                Select {\tt Anywhere} and subsequently select {\tt Allow From Anywhere} in the drop down warning.

            \item Close the {\tt System Preferences} window.
            \item Install HotSpotter and run it.
            \item To re-enable security after running HotSpotter once,
                repeat the above changes to your preferences, except click on
                \texttt{Mac App Store and identified developers}.
	\end{enumerate}
